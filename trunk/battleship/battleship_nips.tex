\documentclass{article}
\usepackage{nips11submit_e, times}

\usepackage{microtype}
\usepackage{amsmath}
\usepackage{amssymb}
\usepackage{usbib}
\usepackage{nicefrac}
\usepackage{url}
\usepackage{auto-pst-pdf}
\usepackage{psfrag}

\usepackage{lmodern}
\usepackage[T1]{fontenc}
\usepackage{slantsc}
\let\scitdefault\scsldefault

\usepackage[algoruled]{algorithm2e}
\SetCommentSty{\normalfont}
\linesnotnumbered 
\dontprintsemicolon

\bibliographystyle{myusmeg-a}
\renewcommand{\citenamefont}[1]{\textsc{\MakeLowercase{#1}}}
\renewcommand{\bibnamefont}[1]{\textsc{#1}}
\renewcommand*{\bibname}{biblography}

\newcommand{\bm}[1]{\mathbf{#1}}
\newcommand{\cm}[1]{\mathcal{#1}}
\newcommand{\data}{\cm{D}}
\newcommand{\given}{\mid}
\newcommand{\intd}[1]{\,\mathrm{d}#1}
\newcommand{\deq}{\triangleq}
\newcommand{\R}{\mathbb{R}}

\newcommand{\psff}[1]{\input{#1.tex}\includegraphics{#1.eps}}

\DeclareMathOperator{\diag}{diag}

\DeclareMathOperator*{\argmax}{arg\,max}
\DeclareMathOperator*{\argmin}{arg\,min}

\begin{document}

\title{The generalized Battleship problem}
\author{
  Roman Garnett, Yamuna Krishnamurthy, Donghan Wang, Jeff Schneider\\
  Carnegie Mellon University\\
  5000 Forbes Avenue\\
  Pittsburgh, Pennsylvania 15213\\
  \texttt{\{rgarnett, ykrishna, dwang, jschneider\}\\@andrew.cmu.edu}
  \And
  Richard Mann\\
  Uppsala Universitet\\
  Uppsala 751\,06\\
  Sweden\\
  \texttt{rmann@math.uu.se}
}

\maketitle

\begin{abstract}
  In many classification problems, including numerous examples on
  modern large-scale graph datasets, a large quantity of unlabeled
  data are available. The cost of obtaining a label, however, can be
  very expensive, for example when human intervention is required.
  Both the semi-supervised and active learning communities have
  approached such problems---the former focuses on aiding the
  classification task by exploiting the distribution of unlabeled
  data, and the latter addresses the problem of requesting the most
  useful labels to minimize the cost spent pursuing the learning goal.
  Here we focus on a specific active binary-classification problem,
  where the goal is to find the members of a particular class as
  quickly as possible.  In some situations, such as fraud detection or
  the investigative analysis of criminal social networks, only the
  members of the malicious class are sought, whereas obtaining labels
  for points in the benign class is only useful for the purpose of
  facilitating that task.  We derive the Bayesian optimal policy for
  this decision problem, as well as myopic approximations for this
  policy, and test our proposed algorithm on two large-scale graph
  datasets.  The optimal policy can be implemented in parallel, and it
  is our hope that the described algorithm can scale to even larger
  problems.
\end{abstract}

\section{Introduction}

In many real-world binary classification scenarios, it can be much
easier to collect input data than it is to observe an associated
label, which could require a relatively expensive human action.  For
this reason, considerable research in semi-supervised and active
learning has considered how to construct useful probabilistic models
when unlabeled data are available, as well as how to solve the
associated decision problem of determining which labels are the most
useful to observe for the chosen learning objective.

Active learning research has often focused on the problem of obtaining
those labels that will improve overall classification accuracy as much
as possible.  Here, we consider a different problem, where the members
of one particular class are deemed critical and are to be located as
quickly as possible.  An example application of this sort is locating
malicious actors in a social or computer network, where finding the
interesting points is more important than secondary measures such as
classification accuracy.  We call this problem the \emph{generalized
  Battleship} problem, in reference to the well-known board game,
which has gameplay that takes the same form.

We will treat this problem in the context of Bayesian decision theory
and derive the Bayesian optimal policy for an appropriate utility
function.  The decision analysis does not depend on the nature of the
underlying classification problem; in particular, the input space and
associated probabilistic model are not constrained.  In this
manuscript, however, we will highlight graph learning as an
application, where problems such as link prediction in social networks
could be potential applications.

The rest of this paper is arranged as follows.  In Sections
\ref{problem} and \ref{related}, we formally describe the generalized
Battleship problem and related work.  In Section \ref{optimal}, we
provide the Bayesian optimal policy for this problem and analyze its
complexity.  Finally, we will present the results of evaluating our
algorithm on two large real-world graph datasets.

\section{Problem Definition}
\label{problem}

Suppose we have a finite set of elements $\cm{X} \deq \lbrace x_i
\rbrace$ and a subset $\cm{R} \subset\cm{X}$, the members of which we
will call \emph{targets.}  We consider the following problem.  Suppose
we do not know which members of $\cm{X}$ are targets \emph{a priori},
but can successively request the binary $\lbrace 0, 1 \rbrace$
response to questions of the form ``is $x$ in $\cm{R}$?'' for
arbitrarily selected elements $x$. We wish to actively select a
sequence of queries $\lbrace q_1, q_2, \dotsc \rbrace$ with the goal
of maximizing the number of positive responses.  This quantity will
serve the role of our utility function.

We will approach this task using Bayesian decision theory.  This will
require selecting a classification model that provides the posterior
probability of a point $x$ belonging to $\cm{R}$ conditioned on
previously observed data $\data$, $p(x \in \cm{R} \given x, \data)$.
It will be mathematically convenient for us to define this problem
equivalently in terms of a binary class label $y$, which we will
define to be the value of the characteristic function $\chi(x \in
\cm{R})$.  In the applications we consider, there will often be a
notion that members of $\cm{R}$ should cluster together under some
measure of similarity on $\cm{X}$, which will influence the choice of
this probabilistic model. 

An atypical real-world problem that could be treated under this
framework is the Milton Bradley board game ``Battleship,'' where two
players alternate firing ``shots'' to determine the presence of a
``ship'' in an identified point of a $10 \times 10$ lattice.  Before
the game begins, each player places five ships on their board, hidden
from the other player, each of which occupies two-to-five contiguous
points on the lattice.  In this case, $\cm{X} = [10] \times [10]$,
$\cm{R}$ identifies the \emph{a priori} unknown locations of the
opponent's ships, and keen players evaluate their beliefs about ship
placement before deciding the locations of their shots.

\section{Related Work}
\label{related}

Active learning is a mature field with a large associated body of
literature \citep{activesurvey}.  The basic premise of active learning
is that if the cost of obtaining a label is very high, then we can
hope to achieve our learning objective with less overall cost by
taking control of the labeling process.  Typically, in the active
binary-classification problem, the chosen learning objective is
related to properties of the associated probabilistic model.  Two
examples include generalization error \citep{zhuactivesemi} and
optimality criteria related to the Fisher information, for example,
\textsc{a}-optimality \citep{activelogistic}.  One of the simplest
active learning techniques for binary classification is
\emph{uncertainty sampling} \citep{uncertaintysampling}, where at any
given point, the algorithm requests the label for the point $x^\ast$
with the greatest posterior uncertainty:
\begin{equation*}
  x^\ast 
  \deq 
  \argmin_x 
  \bigl\lvert 
  p(y = 1 \given x, \data) - \nicefrac{1}{2}
  \bigr\rvert.
\end{equation*}

An intimately related problem to active learning is semi-supervised
learning \citep{semisurvey}, which studies how to use the distribution
of unlabeled data to improve model accuracy.  This is often a critical
problem in active learning due to the high cost of labeling and the
consequent abundance of unlabeled data.

The game of Battleship itself has been subjected to some research as
well.  In \citep{solving}, integer programming models were described
to solve static logic puzzles based on the game.  It has been also
shown that a particular decision problem\footnote{Here ``decision
  problem'' is used in the context of computational complexity.}
related to the game is \textsc{np}-complete \citep{battleshipnp}.

The objective considered in this paper, locating members of an
identified class, is unusual as far as the authors know.  A similar
problem that has been considered is the active discovery of examples
of previously unseen classes \citep{rareclass}.  Additionally, it has
been noted that standard active learning techniques can help uncover
relatively balanced training sets, even with significant imbalance
between class proportion in the data \citep{activeimbalanced}.  This
result suggests that there is promise in active approaches to the
problem we consider, because previously described techniques can
already help to uncover more members of a designated class (targets)
than would be found by random search.  However, our experimental
results show that the formal treatment of the exact problem defined
here performs considerably better than uncertainty sampling.

Here, we will eschew any direct consideration of classification or
model accuracy completely and instead measure the success of our
choice of queries based purely on our outlined goal: the overall
number of targets uncovered by our search.  Of course, accurate models
can facilitate that search, and a natural consequence of our setup is
that the optimal policy will indeed dictate exploring regions with
high uncertainty as needed.

\section{The Optimal Bayesian Policy}
\label{optimal}

As stated above, we will analyze the generalized Battleship problem
from the perspective of Bayesian decision theory and will derive the
optimal policy.  We begin by defining an appropriate utility function
that captures the ultimate goal of the problem.

Define the utility of a set of observations $\data \triangleq
\bigl\lbrace (x_i, y_i) \bigr\rbrace$ to be the number of inputs $x_i$
seen with associated class label $y_i$ equal to 1 (equivalently, the
number of targets found, $\lbrace x_i \rbrace \cap \cm{R}$):
\begin{equation*}
  u(x_1, x_2, \dotsc, y_1, y_2, \dotsc) \triangleq
  \sum_i y_i.
\end{equation*}
This simple expression naturally captures the spirit of the problem
as defined above.

Without loss of generality, we will assume that at the onset we will
only be allowed a fixed number of shots $t$. In applications where the
cost of obtaining a label is high, it is the total cost of the queries
that limits their number, rather than the quantity of unlabeled
points.

We now derive the policy for successively deciding the locations of
our queries.  We begin by considering the case when we are allowed to
make exactly one more query and will then address the general case.

Suppose that we have already made $t-1$ observations $\data_{t-1}$.
To select the location of our final observation $x_t$, we calculate
the expected utility of a candidate point $x_t$, marginalizing out the
unknown value of $y_t$:
\begin{align*}
  \mathbb{E}
  \bigl[
    u(x_t, y_t, \data_{t-1}) \given x_t, \data_{t-1} 
  \bigr]
  &=
  \sum_{y_t \in \lbrace 0, 1 \rbrace} u(x_t, y_t, \data_{t-1}) p(y_t \given x_t, \data_{t-1})
  \\
  &=
  u(\data_{t-1}) + p(y_t = 1 \given x_t, \data_{t-1});
\end{align*}
the optimal decision $x_t^\ast$ is therefore simply the point with the
highest posterior probability of being a target:
\begin{equation}\label{onestep}
  x_t^\ast 
  \triangleq 
  \argmax_{x \notin \data_{t-1}} 
  p(y = 1\given x, \data_{t-1}).
\end{equation}
This fact makes intuitive sense: when we have only one query
remaining, there is no point in considering any potential future
strategic value in its location; rather, we are compelled to greedily
seek one final target.

Given the optimal policy for selecting $x_t$, we now consider the
problem of choosing the location of the second-to-last point
$x_{t-1}$.  While making our decision in this case (as well as with
any other $x_i$ with $i < t$), the problem becomes more difficult
because we must now contemplate the possible consequences of our
choices and how they will impact our future decisions. The mechanical
manifestation of this remark is that during the calculation of the
expected utility for the two-step-lookahead case, we must integrate
out the unknown location of the final observation $x_t$, as well as
its label:
\begin{align}\label{twostep}
  \mathbb{E}\bigl[
    u(x_{t-1}, &y_{t-1}, x_t, y_t, \data_{t-2}) 
    \given x_{t-1}, \data_{t-2}
  \bigr]
  {}
  \nonumber
  \\
  &
  \mspace{-25mu}
  =
  u(\data_{t-2}) 
  + 
  \int 
  u(x_{t-1}, y_{t-1}, x_t, y_t)
  p(y_{t-1} \given x_{t-1}, \data_{t-2})
  \times
  {}
  \nonumber
  \\
  &
  \mspace{225mu}
  {}
  \times
  p(x_t \given \data_{t-1})
  p(y_t \given x_t, \data_{t-1})
  \intd{y_{t-1}} 
  \intd{x_t} 
  \intd{y_t}
  \nonumber
  \\
  &
  \mspace{-25mu}
  =
  u(\data_{t-2})
  +
  2
  p(y_t^\ast = 1 \given x_t^\ast, y_{t-1} = 1, \data_{t-1})
  p(y_{t-1} = 1 \given x_{t-1}, \data_{t-2})
  \nonumber
  \\
  &
  {}
  \mspace{-25mu}
  \phantom{{} = u(\data_{t-2})}
  +
  \phantom{2}
  p(y_t^\ast = 0 \given x_t^\ast, y_{t-1} = 1, \data_{t-1})
  p(y_{t-1} = 1 \given x_{t-1}, \data_{t-2})
  \nonumber
  \\
  &
  {}
  \mspace{-25mu}
  \phantom{{} = u(\data_{t-2})}
  +
  \phantom{2}
  p(y_t^\ast = 1 \given x_t^\ast, y_{t-1} = 0, \data_{t-1})
  p(y_{t-1} = 0 \given x_{t-1}, \data_{t-2})
  \nonumber
  \\
  &
  \mspace{-25mu}
  =
  u(\data_{t-2})
  +
  \Bigl(
    \mathbb{E}
    \bigl[
      u(x_t^\ast, y_t^\ast, \data_{t-1}) \given y_{t-1} = 1, \data_{t-1}
    \bigr] 
    + 1
  \Bigr)
  p(y_{t-1} = 1 \given x_{t-1}, \data_{t-2})
  \nonumber
  \\
  &
  \mspace{62mu}
  +
  \phantom{\Bigl(}
  \mathbb{E}
  \bigl[
    u(x_t^\ast, y_t^\ast, \data_{t-1}) \given y_{t-1} = 0, \data_{t-1} 
  \bigr]
  p(y_{t-1} = 0 \given x_{t-1}, \data_{t-2}),
\end{align}
where the integral over $x_t$ was evaluated trivially because $p(x_t
\given \data_{t-1})$ is simply $\delta(x_t -
x_t^\ast)$,\footnote{Notice that the value of $x_t^\ast$ depends on
  the data $\data_{t-1}$, including the unknown value of $y_{t-1}$; it
  might differ as a function of that conditioning.} where $\delta$
is the Dirac delta function.

To evaluate the two-step expected utility at a point $x_{t-1}$, we
therefore sample over the unknown value $y_{t-1} \in \lbrace 0, 1
\rbrace$; for each possible value of $y_{t-1}$, we find the optimal
last observation $x_t^\ast$ given that fictitious observation as
described above.  There are four possibilities for the possible
realizations of $(y_{t-1}, y_t^\ast)$.  Two of these increase our
current utility by one, and one increases the utility by two, giving
rise to \eqref{twostep}. Note that sampling over $y_t^\ast$ is not
required, because the expected utility in the one-step case has
already been evaluated.  Of course, to select $x_{t-1}$, we again
choose the currently unlabeled point with the highest expected
utility.

To handle $x_{i}$ for $i < (t - 1)$, we simply repeat the procedure
described above recursively.  Pseudocode for the procedure with any
number of remaining shots is shown in Algorithm \ref{lookahead}.

\begin{algorithm}
  
  \SetKwData{utility}{utility}
  \SetKwFunction{findoptimalaction}{{\normalfont \textsc{find-optimal-action}}}
  \SetKwInOut{input}{input}
  \SetKwInOut{output}{output}

  \input{data $\data \deq (\bf{x}, \bf{y})$, 
    allowed number of shots $\ell$, 
    model \hspace*{0.01em} $p(y = 1 \given x,\data)$} 
  \output{the Bayesian optimal action 
    $x^\ast \in \cm{X} \setminus \bf{x}$}

  \BlankLine

  \eIf{$\ell = 1$} {
    $x^\ast \leftarrow 
    \displaystyle 
    \max_{x \in \cm{X} \setminus \data} 
    p(y = 1 \given x, \data)$ \;
    \BlankLine
    \Return $x^\ast$ \;
  }{
    \For{$x \in \cm{X} \setminus \bf{x}$} {
      \utility{$x$} ${} \leftarrow 0$ \;
      \For{$y \in \lbrace 0, 1 \rbrace$} {
        \utility{$x$} ${} \leftarrow$ 
        \utility{$x$} ${} + y + {}$ 
        \findoptimalaction{
          $\data \cup \lbrace x, y \rbrace$, $\ell - 1$\,
        } \;
      }
    }
  }
  \BlankLine
  $x^\ast \leftarrow 
  \displaystyle 
  \max_{x \in \cm{X} \setminus \data}$ \utility{$x$}\;
  \BlankLine
  \Return $x^\ast$ \;
  
  \caption{Function \textsc{find-optimal-action} \label{lookahead}}
\end{algorithm}

%\subsection{Running Time}
%\label{runningtime}

The running time of Algorithm \ref{lookahead} may be calculated
directly.  Suppose there are $n$ points to choose from at time $t -
\ell$.  Without any heuristics, performing exact $\ell$-step lookahead
requires considering all of the
\begin{equation*}
  \prod_{i = 0}^{\ell-1} (n - i)
\end{equation*}
possible choices for the remaining $\ell$ points. For each of these
choices we must also sample $2^{\ell - 1}$ possible observations of
$y_{t-\ell + 1}, \dotsc y_{t-1}$, as well as incrementally condition
the model based on $\ell - 1$ fictitious observations.  If that
conditioning takes time $c$, then the total work required is
\begin{equation*}
  \frac{2^{\ell - 1} (\ell-1)c n!}{(n - \ell)!} 
  = 
  \cm{O}\bigl(c\ell(2n)^\ell\bigr).
\end{equation*}
For lookahead more than a few steps into the future, this procedure
can become daunting due to the sampling required.  This is a common
issue in fully optimal sequential Bayesian decision problems.

One possible and common mechanism for addressing this problem is
instead approximating exact inference by shortening our lookahead
horizon \citep{ego, gpgo}.  For timestep $t - m$ with $m > \ell$, we
may myopically pretend that there are in fact only $\ell$ observations
remaining and choose $x_{t-m}$ by maximizing the $\ell$-step-lookahead
expected utility.  In nontrivial applications, we can often expect
that our uncertainty about the future will increase dramatically with
the number of steps we look ahead. In this case, such a myopic
approach should be a reasonable approximation to the optimal policy,
because this large uncertainty results in little gain from
marginalizing distant future decisions.  In Bayesian Gaussian process
global optimization, a problem with a similar form, a one-step
approximation is often used with great success \citep{ego}.  In
\citep{gpgo}, the authors later described the optimal
$\ell$-step-lookahead procedure and showed that two-step lookahead
can outperform the one-step approximation.

An alternative or potentially complementary approach is to restrict
the search space as we look ahead.  For example, suppose that our
model giving $p(y = 1\given x, \data)$ only considers those
observations in $\data$ that are within the $k$-nearest neighbors of
$x$ under some chosen similarity measure.  In this case, the work
required to perform $\ell$-step lookahead reduces significantly.  For
example, we may perform two-step lookahead by precomputing the
expected utility for one-step lookahead at every point.  Now for a
chosen candidate $x_{t-1}$, when sampling over $y_{t-1}$, we only need
to condition the models for those points $x$ that contain $x_{t-1}$ as
one of their nearest neighbors.  Finding $x_t^\ast$ given $y_{t-1}$
requires much less work, because for $k \ll n$, most of the available
choices will already have the required one-step lookahead utilities
precomputed.  The average-case running time in this scenario now
reduces to approximately $\cm{O}\bigl(c\ell(2k)^\ell\bigr),$ which is
much more manageable and allows for increasing $\ell$ beyond what was
feasible before.  Restricting to $k$-\textsc{nn} models also allows
the algorithm to be easily parallelized and applied to very large
datasets---the calculation of expected utilities is embarrassingly
parallel, and restricting to local models makes each individual
computation relatively fast.

\section{Results}

To evaluate the policy described in Section \ref{optimal}, we created
two binary classification problems on real-world graphs.  In the
applications we have envisioned, the target class will often be a
small fraction of the overall dataset, and we kept this in mind during
the creation of these experiments.

\subsection{Implementation and setup}

We implemented Algorithm \ref{lookahead} recursively in
\textsc{matlab}.  For the classifier $p(y = 1 \given x, \data)$, we
chose to use a simple $k$-nearest-neighbor classifier, where $k =
200$.

We chose to use the simple $k$-\textsc{nn} classifier for several
reasons.  First, our goal in these experiments was to evaluate the
decision algorithm presented above, and we did not want that
evaluation to be confounded by any unrelated details of the
classification mechanism.  Secondly, our desire is that the methods in
this paper can scale to very large graphs, and $k$-\textsc{nn} models
allow for easy parallelism, as well as cost savings as described
above.  We also note that on some similar tasks, such as collaborative
filtering, $k$-\textsc{nn} classifiers have achieved better success
than more complex \textsc{svm} techniques \citep{knnvssvm}.  Indeed,
in preliminary experiments on the datasets described below, several
tested \textsc{svm}s performed worse than the $k$-\textsc{nn}
algorithm used here.

\subsection{CiteSeer\textsuperscript{\textbf{x}} data}

For our first experiment, we created a graph from a subset of the
CiteSeer\textsuperscript{x}\,citation network.  Papers in the database
were grouped based on their venue of publication (after extensive data
cleaning), and papers from the 48 venues with the most associated
publications were retained.  The graph was defined by having these
papers as its nodes (42\,133 in total) and directed citation
relations as its edges.

For this experiment, we designated all papers appearing in
\textsc{nips} proceedings (2\,198 in total, 5.2\% of the dataset) as
targets.  Locating these papers is a difficult task---among the top 10
most-frequent venues in the dataset are the highly related venues
\textsc{aaai}, \textsc{ijcai}, and \textsc{icml}, which together
comprise 15.2\% of the dataset, and there are more venues related to
machine learning in the remaining 38 as well.

\subsubsection{Choosing the $k$-nearest neighbors}

As mentioned previously, a natural assumption to make when
constructing the classifier $p(y = 1 \given x, \data)$ is that the
target class clusters together under some metric or similarity
measure.  For this example, we follow the conclusions of many previous
research efforts\footnote{Dozens of references may be found in the
  impressively complete review by \citet{fousssigmoid}.} and use of
similarity measures based on random walks, which have performed well
both in our experiments and numerous other graph-learning problems
such as collaborative filtering and web clustering
\citep{fousssigmoid}.  In particular, we used the \textsc{granch}
algorithm, which approximates the \emph{truncated commute time}
between two nodes in a directed graph \citep{granch}.  The commute
time is defined in terms of simple Markovian random walks on the graph
and is described below.  \textsc{Granch} can be computed
quickly enough to scale to very large graphs.\footnote{For a graph
  with approximately $1.7 \times 10^5$ nodes and $2.8 \times 10^6$
  edges, we were able to complete a query for the 200 nearest
  neighbors to a node in the graph in 0.37\,s on a machine using a
  single 2.3\,GHz \textsc{amd} Opteron processor core.}

Let $v$ be an arbitrary node in a graph $G \deq (V, E)$.  Consider the
following simple Markovian random walk on the edges of $E$.  At step
$i = 0$, we begin at $v$.  We choose an outgoing edge $e$ incident to
$v$ randomly with probability proportional to its weight and ``walk''
to the other vertex incident with $e$.  At step $i = 1$, we take
another step of the walk by repeating this procedure from the new
vertex and repeat \emph{ad infinitum}.  It is clear that the edge
weights induce a probability distribution on the set of infinite
random walks from $v$, as well as those walks beginning from any other
vertex.  Given a second node $v' \in V$, define the \emph{hitting
  time} from $v$ to $v'$, $h(v, v')$ to be the expected time for a
random walk from $v$ to enter $v'$ for the first time.  Given this, we
may define a symmetrized measure called \emph{commute time}, $c(v,
v')$, which is the expected number of steps a random walk from $v$
takes to hit $v'$ and then return.  It is clear that $c(v, v') = h(v,
v') + h(v', v)$.  Commute time is a metric on $V$, as is its square
root \citep{fouss}.

The truncated commute time constrains the length of random walks to be
of a given finite length. \citet{granch} argue that not only does the
truncation of random walks allow for computational feasibility, but
also that it helps overcome occasional issues with commute time.  For
example, when using standard full commute times, nodes with very large
degree can be ``close'' to a node of interest, even if all paths to
that node are relatively long.  This phenomenon has been noted
recently and described in a theoretical context for random geometric
graphs \citep{commutetimedegree}.  Truncating the walks helps to
discourage this phenomenon.  For the experiments below, we set this
truncation parameter $T$ to 10.

\subsubsection{Experimental details}

We ran 1\,000 steps of both the one-step- and two-step-lookahead
versions of the algorithm presented above.  To facilitate computation,
not every page was considered in the two-step algorithm; rather, we
only calculated the two-step expected likelihood for a subset of the
unlabeled pages.  This subset was chosen to include the unlabeled
pages among the 200 nearest neighbors of the last 10 targets found,
the pages with the last 10 targets found as nearest neighbors, and
additionally a randomly selected subset of the remaining nodes
(comprising 1\% of the remainder).\footnote{This modification was only
  necessary because the \textsc{matlab} implementation was very
  inefficient.}  This heuristically chosen subset was designed to
attempt to enable both exploitation and exploration.  Although the
Bayesian optimal action might not be found among these points, we hope
that at least one point will be ``close enough'' to the optimal action
in expected utility that the overall performance will not be
dramatically impacted.

We also compared our derived policy with a more typical active
binary-classification algorithm (uncertainty sampling), where at each
step, the point with the maximum uncertainty was queried.

Each algorithm was initialized with the label of a single randomly
selected \textsc{nips} paper (which was the same for each algorithm);
all other papers were left unlabeled at the start of the experiment.

At completion, the one-step algorithm had found 226 \textsc{nips}
papers, the two-step algorithm had found 251, and the uncertainty
sampling algorithm had found 206.  Random search would find an
expected number of only 52 papers given the same number of shots.
Figure \ref{citeseerresults} shows the cumulative number of targets
found by each of the methods throughout the run of the experiment.

\begin{figure}
  \centering
  \psff{fig}
  \caption{Cumulative number of targets found during 1\,000 steps of
    several active querying schemes on the CiteSeer\textsuperscript{x}
    data.
  }
  \label{citeseerresults}
\end{figure}

These results support the effectiveness of Algorithm \ref{lookahead}.
The two-step-lookahead procedure was able to find 11.4\% of the
targets after scanning only 2.2\% of the data, five times better than
would be expected by random search.  The two-step-lookahead policy
also outperformed the greedy one-step policy by a significant margin.

\subsection{Wikipedia data}

We ran an additional experiment on a much larger dataset obtained from
Wikipedia.  The input space was Wikipedia pages extracted from the
Computer Science category, as well as all its subcategories.  In
total, there were 171\,236 documents in this dataset.  Pages in the
Programming Languages category or its subcategories (5\,422 in total,
3.17\% of the dataset) were marked as targets.

We repeated the same experiment described above for the
CiteSeer\textsuperscript{x} data on this dataset, except we doubled
the number of shots to 2\,000.  Out of these, the one-step lookahead
procedure located 1\,783 targets.

We interpret this result in two ways.  The impressive performance of
the greedy one-step-lookahead procedure on this task implies that
perhaps the chosen classification problem was too
easy---programming-language pages might simply cluster together
strongly.  On the other hand, we also note that \textsc{granch} was
highly effective in identifying this cluster, empirically confirming
our assertion that it is useful for this purpose. In light of these
observations, we modified the problem setup to make it more difficult,
while still providing a means of testing our search policy.  Rather
than analyze the link graph of the Wikipedia pages, we used only their
textual content for discriminating between classes.

To approach this modified problem, we trained a latent Dirichlet
analysis (\textsc{lda}) topic model from these documents using the
\textsc{mallet} software package \citep{mallet}. The number of topics
was set to 200.  The default parameters for \textsc{mallet} were used,
except the hyperparameters of the \textsc{lda} model were optimized
during the learning process.  Once the topic vector for each document
was found, we measured the cosine similarity between these topic
vectors and recorded the 200 most similar documents for each Wikipedia
page, as well as the associated cosine similarities, which were used
to weight the $k$-\textsc{nn} algorithm.

We ran 2\,000 steps of the one-step and two-step optimal algorithm, as
well as uncertainty sampling, given this problem setup. Each algorithm
was initialized with the label of a single page in the Programming
Language category (which was the same for each algorithm); all other
pages were left unlabeled at the start of the experiment.

Among the 2\,000 nodes selected for querying by the one-step-lookahead
algorithm, 975 were in the Programming Languages category.  The
two-step-lookahead algorithm found 990, and uncertainty sampling
discovered 734. In contrast, random search would be expected to find
only 63 targets with the same number of evaluations.  The
two-step-lookahead algorithm was able to locate 18.2\% of the targets
after querying only 1.17\% of the dataset.  These results again
confirm the effectiveness of Algorithm \ref{lookahead}.

\subsection{Discussion}

In light of these results, we make an observation on the nature of the
three algorithms tested.  First, both the relatively good quality of
uncertainty sampling, as well as its relative inferiority to the
algorithm described above, can be explained by its slightly different
goal.  Querying the point with the maximum uncertainty causes the
classification boundary to be explored.  Although this finds many
targets (because they lie on the boundary just as often as the
others), the different utility function driving the standard algorithm
causes it to also query nodes less likely to be targets more often.

In a sense then, uncertainty sampling is a purely explorative
approach.  The one-step optimal policy, on the other hand, is a greedy
and purely exploitative approach, and it is not surprising that it
performs better than uncertainty sampling.  We believe the increased
performance of the two-step-lookahead policy is that by looking two
(or more) steps into the future, the algorithm can consider the
relative tradeoffs of exploration versus exploitation, which are two
goals that must typically be simultaneously be satisfied in
optimization procedures.

\section{Conclusion and Future Work}

We have introduced the problem we call ``generalized Battleship,''
that creates the challenge of actively seeking out members of an
identified class.  We defined a natural utility function and derived
the Bayesian optimal policy for the associated decision problem.

The policy presented was tested empirically on two real-world graph
datasets and performed well in comparison to a common active learning
technique for binary classification, and we believe this difference is
due to the different goals of the two methods.

In the future, we hope to investigate different approaches to the
underlying classification mechanism, particularly for graphs.  It
would also be interesting to investigate other approaches for
increasing the depth of lookahead that can be considered by our
method; for example, perhaps a branch-and-bound heuristic can be
employed with success.

\bibliography{battleship}

\end{document}
